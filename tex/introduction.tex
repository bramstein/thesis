\chapter{Introduction}
\section{Background}
The FIRST system (Facts on International Relations and Security Trends) is a free-of-charge system for politicians, journalists, researchers and the interested public\footnote{Facts on International Relations and Security Trends website: \url{http://first.sipri.org/}}. The main seven user groups are diplomats, decision makers, researchers, journalists, students, administrative personnel and librarians. FIRST is a joint project of the International Relations and Security Network (ISN) and the Stockholm International Peace Research Institute (SIPRI).

The FIRST system contains documented information from research institutes around the world. It covers areas in the field of international relations and security, such as hard facts on armed conflicts and peace keeping, arms production and trade, military expenditure, armed forces and conventional weapons holding, nuclear weapons, chronology, statistics and other reference data.

The data are stored in roughly 50 different datasets. These datasets range from relational databases to HTML or XML generated by partner institutes in other geographic locations. The FIRST system provides an easy to use web-based interface and offers transparent search methods for all of these datasets.

A new version of the FIRST system has been under development since early 2007. It features a completely redesigned data back end based on the REST architecture \cite{fielding00}, support for multiple languages, and a new dataset plug in system. It also exports data and meta-data in the Resource Description Format (RDF) \cite{rdf04} and various other formats generated from RDF (such as XHTML and JavaScript Object Notation \cite{crockford06}.) The user interface built on this new system is however heavily influenced by the user interface of the old system, and thus unable to showcase the full power of the new back end. A new user interface currently in the design phase calls for the development of a data visualization library with support various types of charts. This thesis project involves the design and implementation of such a library.

\section{Goals \& Limitations}
The goal for this thesis project is to create a chart library prototype which meets the following requirements:
\begin{itemize}
\item the library should run client side, using JavaScript
\item visualizations are implemented using the HTML5 Canvas element
\item the input for charts should be either a XHTML table, or a JSON object
\item the following chart types must be supported: bar charts, line charts, and scatter plots
\end{itemize}
Beyond these technical requirements, the visualizations should also be visually pleasing and meet the guidelines set out in the book "Show Me the Numbers, Designing Tables and Graphs to Enlighten" \cite{few06}.

The prototype only has to work under controlled circumstances, i.e. targeting a specific browser or operating system (preferably Mozilla Firefox) is acceptable.

\section{Overview}
This thesis report consists of two main parts, which closely correspond to the structure of the chart library. The second chapter is a discussion of the architecture and design of the core components of the library. The third chapter is a discussion and showcase of the chart "plugins" that are built using the core components. These plugins are independent of each other and built using the public interface of the core library.

The fourth chapter contains an overview of other open source web visualization and chart libraries writting in JavaScript. The fifth chapter discusses future enhancements to the library. The sixth and last chapter contains the pro's and con's of the library and other concluding thoughts.


