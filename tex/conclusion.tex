\chapter{Conclusion}

One of the main advantages of the chart library is its component based nature. All of the components such as titles, legends, and even charts themselves can be moved around easily and laid out with a few lines of code. This makes the library very easy to modify and maintain. New components can be developed without any changes to existing components and---as long as they meet the simple component interface requirements---can be integrated and laid out just as easily. The use of layout managers also proved to be very useful as it separates the layout algorithms from the components.

Another advantage is the plug-in structure, which enables plug-in authors to easily add new chart types. Authors will have all the standard components at their disposal, and basically only have declare which input they accept and then iterate through the data and draw the chart.

Firefox, Opera and Safari all support various subsets of the HTML5 Canvas element. Fortunately there is a common subset of functionality that is sufficient for use in the chart library (apart from a text API.) Unfortunately, the only browser that does not support the HTML5 Canvas element is also the browser with the largest market share: Internet Explorer. There are however JavaScript and ActiveX implementations available that provide a HTML5 Canvas element API for Internet Explorer and convert the API calls into Vector Markup Language (VML) elements \cite{google08}. By using one of these implementations, the chart library supports all major browsers without modification to the source code.

Whether or not the choice of using the HTML5 Canvas element for drawing charts is correct remains to be seen. Both SVG and Canvas currently have problems with browser support and interoperability (either something is not supported or implementations differ.) Currently the only safe choice would be to develop for both platforms until a clear winner has emerged (if there is such a thing.) The graphics module makes it easy to support both drawing APIs as it encapsulates and abstracts all drawing calls, shielding the rest of the chart library from "platform" specific APIs. Alternatively, SVG could be used for static charts, while Canvas is used for dynamic charts. The SVG structure makes it easy to create static charts, while Canvas, with its low level API, makes animation and performance easier to achieve.

Compared to other chart libraries, features that stand out are the aforementioned component structure, accurate function plotting, crisp line drawing, and the plug-in extension mechanism. The current code base is modular and easily extended, thus making it a good platform for further development. The library is also independent of other JavaScript frameworks, unlike other chart libraries which depend on libraries such as jQuery, Prototype and MooTools.


