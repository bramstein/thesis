\section{Axis}
The axis module is a factory method for creating an object representation of a chart axis. The resulting object can represent two different kinds of axes: a numeric axis or a categoric axis. Axis themselves are not components that can be used by a layout manager. Instead they are parameters to the canvas module, which draws them (and is a component that can be laid out.) 

To create a numeric axis, the user supplies the axis module with an interval indicating the range the axis should span. The user can also optionally supply either an array of minor and major tick mark locations, or the desired number of minor and major tick marks.  When a number is supplied the axis module automatically calculates the tick locations in a visually pleasing manor, implemented using an algorithm described in the "Graphics Programming Gems" series \cite{heckbert90}. 

The algorithm boils down to choosing tick mark locations that have "nice" round numbers starting with either 1, 2, or 5. The algorithm favours these numbers over others. If such numbers are not desired, the user could always supply the axis module with custom values instead of having them automatically calculated based on the interval.

A categoric axis can be created by supplying the axis module with an array of labels. The axis module will then create an axis with an empty interval and use the supplied categories as labels.

All axes can also have a custom set of labels for the tick marks, and a label for the axis itself.
