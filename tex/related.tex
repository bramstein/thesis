\chapter{Related work}
This survey of related work focuses on open-source visualization toolkits which are based on non-proprietary platforms. The multitude of Flash based toolkits and libraries are thus not taken into consideration. The survey is further limited to chart libraries that draw on the client side as that is one of the requirements of this thesis project.

\section*{Flot}
Flot is a pure JavaScript plotting library for use with the jQuery JavaScript library. Its goals are ease of use, attractive looks and user interaction such as zooming and mouse tracking\footnote{Flot: \url{http://code.google.com/p/flot/}, jQuery: \url{http://www.jquery.com/}}. It is built upon the HTML5 Canvas element and has support for line, bar, and point charts. For text drawing it uses absolute positioned \code{div} elements. 

\section*{PlotKit}
PlotKit is a JavaScript plotting library for the MochiKit library with a focus on time series\footnote{PlotKit: \url{http://www.liquidx.net/plotkit/}, MochiKit: \url{http://mochikit.com/}}. It has support for bar, line and pie charts. There are several renderers available with support for both the HTML5 Canvas element and SVG. Additionally it provides renderers for basic charts and charts with extra decorations (borders, shadows, etcetera.) PlotKit also draws text labels using absolute positioned \code{div} elements.

\section*{Flotr}
Flotr\footnote{Flotr: \url{http://solutoire.com/flotr/}, Prototype: \url{http://prototypejs.org/}} is a JavaScript plotting library which draws great inspiration from the Flot library mentioned earlier. It has similar features, but this time it is built for the Prototype JavaScript library. In contrast to the Flot library, it has support for negative values, styling graphs via CSS, and several kinds of events for user interaction.

\section*{ProtoChart}
ProtoChart is yet another Prototype based chart library\footnote{ProtoChart: \url{http://www.deensoft.com/lab/protochart/index.php}, Prototype: \url{http://prototypejs.org/}}, inspired by the Flot, Flotr and PlotKit libraries. It features line, bar, pie, curve, and area charts. ProtoChart also supports interaction, and combining different chart types into one chart.

\section*{Bluff}
Bluff is a port of the Gruff Charting library written in the Ruby programming language\footnote{Bluff: \url{http://bluff.jcoglan.com/}, Gruff: \url{http://nubyonrails.com/pages/gruff}}. It has support for line, bars, pie, area, and spider charts. It uses the HTML5 Canvas element for drawing the charts and overlays absolute positioned \code{div} elements for labels and legends. A distinct feature for Bluff is that it supports (simple) HTML tables as input for charts.

\section*{FGCharting}
The FGCharting library is a demonstration of how accessible charts can be made by parsing HTML tables using the jQuery library\footnote{FGCharting: \url{http://www.filamentgroup.com/lab/creating\_accessible\_charts\_using\_canvas\_and\_jquery/}}. It supports bar, pie, line and area charts, and renders these using the HTML5 Canvas element. Again absolute positioned \code{div} elements are used for labels, titles and legends.
